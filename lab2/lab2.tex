\documentclass[12pt,a4paper]{article}

% \newcommand*{\TypeChinese}{} % Chinese support
\newcommand*{\AdvancedDocument}{} % include code and math
\newcommand*{\WithHeader}{}

% basic packages
\usepackage[margin=2cm,headheight=15pt]{geometry}
\usepackage{graphicx,subfigure,indentfirst,hyperref,colortbl,caption,cite,color,xcolor}
\hypersetup{colorlinks=true,urlcolor=blue,linkcolor=blue}
% for handling underscores
\usepackage{lmodern,relsize}
\usepackage[T1]{fontenc}
\renewcommand{\_}{\textscale{.5}{\textunderscore}}

\ifdefined\AdvancedDocument
	% minted is better than listing.
	\usepackage{minted}
	% it requires minted of a newer version.
	\setminted{linenos=true, frame=lines, framesep=2mm}	
	\usepackage{amsmath,amssymb,bm}
\fi

\ifdefined\TypeChinese
	\usepackage{xeCJK,fontspec}
	\XeTeXlinebreaklocale "zh"
	\XeTeXlinebreakskip = 0pt plus 1pt
	\setmainfont{KaiGen Gothic TW}
	\setCJKmainfont{KaiGen Gothic TW}
	\setmonofont{Droid Sans Mono}
	\renewcommand{\baselinestretch}{1.3}
\fi

\ifdefined\WithHeader
	\usepackage{fancyhdr}
	\fancypagestyle{plain}{
		\fancyhf{}
		\chead{GPU Programming 2018 Spring \textbar ~Dep. of CSIE, NTU}
		\cfoot{\thepage}
		\rfoot{Programming Assignment \#2}
	}
	\pagestyle{plain}
	\renewcommand{\headrulewidth}{1pt}
	\renewcommand{\footrulewidth}{2pt}
\fi

\newcommand{\figref}[1]{Figure \ref{Fig:#1}.}
\newcommand{\tabref}[1]{Table \ref{Tab:#1}.}
\newcommand{\lstref}[1]{Listing \ref{Lst:#1}.}
% \graphicspath{{fig/}}

\begin{document}
\title{Programming Assignment \#2}
\author{Author: Yu Sheng Lin \and Instructor: Wei Chao Chen}
\maketitle

\section{Goals}

You have to

\begin{enumerate}
\item Learn how to use the Thrust library.
\item Learn how to implement some parallel algorithms on GPU.
\end{enumerate}

in this assignment.

\section{Requirements}

This is a two-part fixed assignment.
You are asked to implement the same functionality in both part.
However, in part I you are only allowed to use existing libraries
while in part II the efficiency of your code is also considered during grading.

In this assignment you are required to count the position
of each character inside the word it belongs to.
\lstref{count} provides a sample input and output.

\begin{listing}[ht]
\begin{minted}{text}
gpu qq  a hello   sonoda (input)
123012001012345000123456 (output)
\end{minted}
\caption{Example: Count the Position in Words.}\label{Lst:count}
\end{listing}

You may easily come up with an $O(n)$ sequential algorithm
and an $O(nk)$ parallel algorithm,
where $n$ is the length of the input and $k$ is the maximum length of a word.

The input is generated in a pseudo-random manner,
while we will use a different random seed during grading.
You can assume that $k=500,~n\approx 4\times10^7$ and
the input only contains characters \verb+[a-z]+
and we use linebreak \verb+'\n'+ as the spaces.

You have to implement a function whose signature is \lstref{count_cpp}.
All pointers are device pointers and \verb+text_size+ is the $n$.
It should also be noticed that \verb+gridDim.x+ cannot exceed $2^{17}=131072$
if you don't use \verb|-arch sm_30| (or higher)
\footnote{We assume that you have a GPU newer than Kepler architecture.} compile flag.

\begin{listing}[ht]
\begin{minted}{cpp}
void CountPosition1(const char *text, int *pos, int text_size);
void CountPosition2(const char *text, int *pos, int text_size);
\end{minted}
\caption{The function signature of part I.}\label{Lst:count_cpp}
\end{listing}

\subsection{Part I: Using the Thrust Library (40pts)}\label{thrust}

Thrust is a useful API including many common parallel computing patterns,
and in this part you should only include
\begin{itemize}
\item Declaration of native and \verb+thrust::*+ types,
\item A few \verb+struct+s with call operator and
\item \verb+thrust::*+ and native CUDA functions like \verb+cudaFree+
      (namely \verb+__global__+ functions are not allowed)
\end{itemize}
in \verb+CountPosition1+.

Here are some hints:
\begin{itemize}
\item You will need the document \url{https://thrust.github.io/doc/modules.html}
\item I have already included some necessary headers.
\item If you write more than 10 lines, then you are probably wrong.
\end{itemize}

\subsection{Part II: Implement Your Own Kernel (60pts+15pts bonus)}

In part II, you have to implement the same functionality from scratch,
and {\color{red}{no external API and library is allowed}}.
We provide some hints about the $O(n\ln k)$ algorithm in a separate PDF
while it's not the only solution,
and you can decide whether to read them by yourself.

According to our experience, when $k=500$, $O(n\ln k)$ and $O(nk)$
might have comparable speed. To achieve the best efficiency and
outperform your classmate, we also challenge you to implement both and
compare them if possible.
According to the execution time of your program,
at most 15pts bonus will be given.

\section{Submission}

\begin{itemize}
\item The submission deadline is 2018/5/10 23:59 (Thur.).
\item The efficiency of part I will NOT be considered during grading, but if you
break the rules listed in part \ref{thrust}, you will get 0pt.
\item The efficiency part II will also be considered during grading, and pure CPU implementations would be disqualified.
\item You can only modify \verb+lab2/counting.cu+, and we will only copy this file from your repo.
\item The compile flags are \verb|--std=c++11 -O2 -arch sm_50|, and the environment is ArchLinux with CUDA 9.1. Although we will test your code on a Linux machine with one GTX 970, you should not use platform dependent libraries.
\item Do not hack the answer by inspecting the memory or modifying the stack. This is a course about GPGPU programming, not assembly programming.
\item Please also refer to assignment \#0 for more details.
\end{itemize}

\end{document}
